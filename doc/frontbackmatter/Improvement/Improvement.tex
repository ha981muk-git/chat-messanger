%******************************************************************************
% Improvement
%******************************************************************************
\chapter{Improvement}\label{ch:improvement}


\section{Database}\label{sec:database}
In our project an array was used to store the data
but if the project needs to be improved, it would be very good and very helpful to use the database,
because the use of databases prevents redundancies \(multiple storage of the same information\) and
inconsistencies \(problems with updating multiple stored records\) in the data.
When using the database, the client information and its messages are stored permanently, so the
client has to register with the server only once, and then he has the possibility to log in.
On the other hand, client messages are automatically stored and always available.

Benefit of Using the DB:
\begin{itemize}
    \item Databases can easily handle multiple users.
    \item Databases are very reliable because they store information accurately.
    \item Databases avoid unnecessary information.
    \item Databases allow powerful and interesting information processing.
    \item Databases are extensible.
    \item Databases can handle large amounts of data
\end{itemize}
\noindent
Even if you don't need to process large amounts of data yourself (yet), databases are useful for
smaller amounts of data. With the capacity to handle so much data, a well-designed database can
serve you for many years as it grows with you.


\section{Application-Programming-Interface}\label{sec:application-programming-interface}
it enables data exchange between the server and the client. The API concept was designed by Roy
Fielding in 1994.
It uses http requests to access information via GET, POST, PUT and DELETE\@.
When using REST interface, a clean separation is created between the backend (server) and the
frontend (client).
In API, (de-) serialization is passed, and it is done by determining the media-type of REST interface.
API has different layers, based on the architecture of the API the layer is determined

\begin{itemize}
    \item Level 0 "The Swamp of POX": the exchange of information is done only through an API\@.
    \item Level 1 "Resources": an API must be implemented for each object.
    \item Level 2 "HTTP Verbs": the http verbs (GET, POST, PUT and DELETE) are implemented each API
    \item Level 3 "Hypermedia Controls": a URL is returned for each child object instead of the object.
\end{itemize}
\noiondent
The disadvantage of layer "0" and "1" is that only the POST request is implemented in the API,
which means you can read, write, edit and delete the data with the same request.
This means that whoever has read permission can also edit data.
The advantage of layer "2" is that with the help of HTTP request can be controlled which right has
this request.
Layer "3" brings a special advantage that the nested-objects are fetched with an API and not packed
in the root object.

\section {Videoconferencing}\label{sec:vidioconferenz}
Video conferencing can be very important in our project in the future, because it has many advantages
for companies and people, for example
1-Video conferencing can easily overcome spatial distances. Location-independent communication
saves valuable time and long journeys, and can thus speed up workflows and decision-making processes
enormously.
2-The video conferencing is very helpful for people working abroad and also for students studying abroad.
so that you can contact with their families.
You can add the ability of screen sharing for video conferencing, so that you can have multiple applications
in one application.


\section{Validation}\label{sec:validation}
Data validation may save the day of computer programmers, whatever programming language they use.
In fact, processing invalid data is a waste of resources at best, and a drama at worst if the problem
remains unnoticed and wrong results are used for business. Answer Set Programming is not an exception,
but the quest for better and better performance resulted in systems that essentially do not validate
data in any way. Even under the simplistic assumption that input and output data are eventually
validated by external tools, invalid data may appear in other portions of the program, and go
undetected until some other module of the designed software suddenly breaks.
an example about validation for our project is
that you use validation for username.
username must contain a minimum of 3 letters and a maximum of 16 letters, if the user enters more
than 16 letters or less than 3 letters, the program must show a pale.



