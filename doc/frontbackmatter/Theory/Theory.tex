%******************************************************************************
% Theory
%******************************************************************************
\chapter{Theory}
     
    \section{Java}
    Java is one of the most popular programming language.It is develop by Oracle Corportation
    in May 23, 1995. It is also used to develop mobile apps, web apps, desktop apps, games and
    much more. It has helped to reduce costs, shortens development timeframes, drive innovation,
    and improve application services.
      \medskip

     \noindent
     There are five primary goals in the creation of the Java language.
     \begin{itemize}
        \item It must be simple, object-oriented, and familiar.
        \item It must be robust and secure.
        \item It must be architecture-neutral and portable.
        \item It must be interpreted, threaded and dynamic.
    \end{itemize}

    \section{Thread}
    A thread of execution is the smallest sequence of programmed instructions that can be managed 
    independently by a scheduler, which is typically a part of the operating system.
    The implementation of threads and processes differs between operating systems,but in most cases
    a thread is a component  of a process. The multiple threads of a given process may be executed
    concurrently \(via multithreading capabilities\), sharing resources such as memory, while different
    processes do not share these resources.
    \medskip

	\noindent
    A thread is a thread of execution in a program. The Java Virtual Machine allows an application to have
    multiple threads of execution running concurrently. Every thread has a priority. Threads with higher 
    priority are executed in preference to threads with lower priority.

    \section{Git}
    Git is an open source distributed version control system, mainly used for source code management, with
    an emphasis on speed. Git was initially designed and created by Linus Torvalds for Linux kernel development.
    Git operates on a decentralized architecture, so every Git working directory is a full-fledged repository
    with a complete history and full revision-tracking capabilities, and is not dependent upon network access or 
    a central server. Unlike popular non-distributed predecessors, such as Subversion and CVS, Git only needs a 
    central server for one thing: publishing changes to users of that server. You can equally share changes 
    directly with other people without the need to consult a central hub.

    \section{Gitlab}
    GitLab Inc. is an open-core company that provides GitLab, a DevOps software package that combines the
    ability to develop,secure, and operate software in a single application. The open source software project
    was created by Ukrainian developer Dmitriy Zaporozhets and Dutch developer Sytse Sijbrandij. Since its
    founding, GitLab Inc. has been centered around remote work. GitLab has an estimated 30 million registered 
    users, with 1 million being active licensed users.

    \section{Project}
    A project is well-defined task, which is a collection of several operations done in order to achieve a goal 
     \(for example, software development and delivery\).
     \medskip
    
    A Project can be characterized as:
    \begin{itemize}
        \item Every project may has a unique and distinct goal.
        \item Project is not routine activity or day-to-day operations.
        \item Project comes with a start time and end time.
        \item Project ends when its goal is achieved hence it is a temporary phase in the lifetime of an organization.
        \item Project needs adequate resources in terms of time, manpower, finance, material and knowledge-bank.
    \end{itemize}

    A software Project is the complete procedure of software development from requirement gathering to testing and
    maintenance, carried out according to the execution methodologies, in a specified period of time to achieve 
    intended software product.

   \section{Our Software  Project}
   In our project, we have developed an application where two or more users can interact and exchange messages 
   of different file types such as text, image, video, PDF, and so on. Mainly there are two GUI windows which make 
   the communication user friendly and easy to communicate. first window is the server and the other is the client. 
   The server window has the main privilege of starting and stopping the application. When the server is started,
    the clients can exchange messages. The client sends a message to the server and the message is forwarded 
    by the server to the target user. The recipient client can receive the message. If necessary, admin can delete 
    the clients or the group by selecting.
    \medskip
    
    \noindent
    On the Client window, user can send messages to different users and groups. First of all, user needs to select user
     or group he/she want to chat with other user or group, user will be able to send text to the other user and group
      in the prototype version of our application.There is small text area where 
user will be able to write in it and if they want to attach other document with it,can add with add File button and 
forward it to other with send button. User can navigate other users and group using the search. User can also 
select new contacts and create new group with users. On the config tab, the ip address, port number and name of 
server and client  can be seen and can be saved if save button is clicked.
   \medskip
   
   \noindent
   Our team divided the development of this application into different versions, with each new version 
   our team will add more fuctionality, features, try to fix bugs and provide user friendly experiece to 
   the users.In the version 1.0, users can simply chat with other user and groups. In the version 2.0, 
users will be able to send images,text files,pdf files etc. In the version 3.0, user can send videos
 and play it on message window. Later, our team will come with more ideas and work with user's 
 suggestions to improve our application.
 
\end{document}




\begin{comment}
Reference 

https://en.wikipedia.org/wiki/Java_(programming_language)
https://www.oracle.com/java/
https://www.w3schools.com/java/
https://en.wikipedia.org/wiki/Thread_(computing)
https://en.wikibooks.org/wiki/Git
https://www.tutorialspoint.com/software_engineering/software_project_management.htm

\end{comment}

    \end{align*}
\end{document}