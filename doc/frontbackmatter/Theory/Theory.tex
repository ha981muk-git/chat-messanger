%******************************************************************************
% Theory
%******************************************************************************
\chapter{Theory}\label{ch:theory}
\section{Java}\label{sec:java}
    Java is one of the most popular programming language.
    It is develop by Oracle Corporation in May 23, 1995.
    It is also used to develop mobile apps, web apps, desktop apps, games and much more.
    It has helped to reduce costs, shortens development timeframes, drive innovation,
    and improve application services.
    \medskip

    \noindent
    There are five primary goals in the creation of the Java language.
    \begin{itemize}
        \item It must be simple, object-oriented, and familiar.
        \item It must be robust and secure.
        \item It must be architecture-neutral and portable.
        \item It must be interpreted, threaded and dynamic.
    \end{itemize}


    \section{Socket}\label{sec:socket}
    Socket programming in Java provides a tool that allows communication between two applications
    that run on different devices.
    Socket allows a connection between a client and a server.
    Typically, it provides a way to connect two nodes on a network to exchange data.
    One node listens on a specific port on an IP, while the other socket connects to the other to
    establish a stable connection.
    \medskip

    \noindent
    Initially, the client must wait for the server to start.
    While the server is running, the client must send a request to the server and wait for its
    response.
    The socket connection uses two computers that have the necessary information about each
    other's location on the network location, i.e.IP address and port.
    \medskip

    \noindent


    \section{Thread}\label{sec:thread}
    A thread of execution is the smallest sequence of programmed instructions that can be managed
    independently by a scheduler, which is typically a part of the operating system.
    The implementation of threads and processes differs between operating systems,but in most cases
    a thread is a component  of a process.
    The multiple threads of a given process may be executed concurrently
    \(via multithreading capabilities\), sharing resources such as memory, while different
    processes do not share these resources.
    \medskip

	\noindent
    A thread is a thread of execution in a program.
    The Java Virtual Machine allows an application to have multiple threads of execution running
    concurrently.
    Every thread has a priority.
    Threads with higher priority are executed in preference to threads with lower priority.

    \section{Git}\label{sec:git}
    Git is an open source distributed version control system, mainly used for source code
    management, with an emphasis on speed.
    Git was initially designed and created by Linus Torvalds for Linux kernel development.
    Git operates on a decentralized architecture, so every Git working directory is a
    full-fledged repository with a complete history and full revision-tracking capabilities,
    and is not dependent upon network access or a central server.
    Unlike popular non-distributed predecessors, such as Subversion and CVS, Git only needs a
    central server for one thing: publishing changes to users of that server.
    You can equally share changes directly with other people without the need to consult a central hub.

    \section{Project}\label{sec:project}
    A project is well-defined task, which is a collection of several operations done in order to
    achieve a goal \(for example, software development and delivery\).
    \medskip
    \noindent
    A Project can be characterized as:
    \begin{itemize}
        \item Every project may has a unique and distinct goal.
        \item Project is not routine activity or day-to-day operations.
        \item Project comes with a start time and end time.
        \item Project ends when its goal is achieved hence it is a temporary phase in the lifetime
                of an organization.
        \item Project needs adequate resources in terms of time, manpower, finance, material and
                knowledge-bank.
    \end{itemize}

    A software Project is the complete procedure of software development from requirement gathering
    to testing and maintenance, carried out according to the execution methodologies,
    in a specified period of time to achieve intended software product.

   \section{Our Software  Project}\label{sec:our-software--project}
    In our project, we have developed an application where two or more users can interact and
    exchange messages of different file types such as text, image, video, PDF, and so on.
    Mainly there are two GUI windows which make the communication user friendly and easy to
    communicate.
    first window is the server and the other is the client.
    The server window has the main privilege of starting and stopping the application.
    When the server is started, the clients can exchange messages.
    The client sends a message to the server and the message is forwarded by the server to the
    target user.
    The recipient client can receive the message.
    If necessary, admin can delete the clients or the group by selecting.
    \medskip
    
    \noindent
    On the Client window, user can send messages to different users and groups.
    First of all, user needs to select user or group he/she want to chat with other user or group,
    user will be able to send text to the other user and group in the prototype version of our
    application.There is small text area where user will be able to write in it and if they want
    to attach other document with it,can add with add File button and forward it to other with send
    button.
    User can navigate other users and group using the search.
    User can also select new contacts and create new group with users.
    On the config tab, the ip address, port number and name of server and client can be seen and
    can be saved if save button is clicked.
    \medskip
   
    \noindent
    Our team divided the development of this application into different versions, with each new
    version our team will add more functionality, features, try to fix bugs and provide user
    friendly experience to the users.In the version 1.0, users can simply chat with other user
    and groups.
    In the version 2.0, users will be able to send images,text files,pdf files etc.
    In the version 3.0, user can send videos and play it on message window.
    Later, our team will come with more ideas and work with user's suggestions to improve our
    application.
    \medskip

    \noindent

    \section{UDP Protocol}\label{sec:udp protocol}
    UDP is a communication protocol implemented across the internet for especially time-sensitive
    transmissions such as video playback or DNS lookups.
    UDP provides a quick transfer of data.
    Unlikely, as any other protocol it is also a way of transferring data between two computers in
    network.
    By UDP, data are send in packets directly to the target IP address and port number,
    without establishing a connection first.
    One computer can simply start transferring data to other.
    If there is any errors, loss, and duplication, the UDP doesn't resend the data.

    \medskip

    \noindent

    \section{Our Protocol Segment Format}\label{sec:our protocol}
    Our  basic protocol segment  consist of opcode, sender and receiver. 
    The bases are divided into different bytes arrays.
    The first 4-bytes of protocol segment is reservered for opcode which whose
     basic functionality is to store  functionality like register,deregister, search, 
     poller, message, create group, join group, leave group and group message.
    The secound chunk of 16 bytes presents IP address of the sender computer. 
    The third chunk of 16 bytes reserved for the receiver IP Address.
    \medskip
    
    \noindent
    To register the client with the server, the client must send the basic protocol, host
     name, and port number to the server. 
    The 0x01 opcode is required to register a client with the server using this protocol.
    The hostname requires 16 bytes of data to be stored as part of the protocol and 
    the port number requires 4 bytes of data.
    To log out, the client  needs to send the opcode 0x02 to server  with the sender and
    recipient IP address.
    Then the server logs off the client.
     \medskip
    
    \noindent
    If the sender wants to send data to the receiver, it must specify the base protocol and type
    the content and the content itself, where the type of data is stored in 4 data bytes and the
    the content can have any length. 
 
    The type of  data has five different type.
    \begin{itemize}
    	\item Message type register is presended by 0x01.
    	\item Message type deregister is given  by 0x02.
    	\item Message type search is reserved as 0x03.
    	\item Message type message and file is given by 0x04 and 0x05 respectively.
    \end{itemize}
	 \medskip
	
	\noindent
	To obtain a list of clients, the sender must use the opcode 0x03 with the IP address of the sender 
	and the recipient and the name of the group.
	To create a group, sender needs to implement base protocal with 0x06 opcode and name 
	of the group.Usually, the hostname and name of group takes 16 bytes of data.
	In the same way, sender can to apply  base protocal with 0x07 opcode and name 
	of the group to join the group.
	By just change opcode of above line to 0x08, sender can easily leave the group.
	
	To send a message to a group, the sender who wants to send data to the group must specify
	 the basic protocol  with opcode 0x09, in the addition of the name of the group of 16 bytes 
	 of data and the type of the content and the content itself, where the data type is stored 
	 in 4 data bytes and where the content can be of any length. 
    \medskip

    \noindent
    
    \section{Improvement in Application}\label{sec:improvement in application}

    \subsection{API-implementation}\label{subsec:api-implementation}
    To further improve our application in the future, we would use an API for frontend,
    backend and database and to separate frontend, backend and database from each other.
    
    \subsection{Usable on multiple devices} \label{subsec:usable-on-multiple-devices}
    The current version of our application can be used on Linux, Mac and Windows.
    In the upcoming version we will bring it to other devices like Android, tablets, etc.

    \subsection{Implementation database}\label{subsec:implementation-database}
    Our team will integrate this application to store data, backup client and server data and 
    reimplementation of the data.

    \subsection{Implementation on higher protocol}\label{subsec:implementation-on-higher-protocol}
    To use it on a higher protocol, our team will apply it with the https protocol. 
    so we can use it in websites and browsers.


	\subsection{validation overall}\label{subsec:validation overall}
	To improve the viability of our application, we will add some validation features to make the application more realistic and error-free.
	And know that users are getting the right data.
	

   \subsection{validate of port number}\label{subsec:validate of port number}
	We would add some functions to confirm and validate the port number on the devices
	that the data is transferred to the right person and device.
   



%   Reference
%    https://en.wikipedia.org/wiki/Java_(programming_language)%   https://www.oracle.com/java/
%    https://www.w3schools.com/java/
%    https://en.wikipedia.org/wiki/Thread_(computing)
%    https://en.wikibooks.org/wiki/Git
%    https://www.tutorialspoint.com/software_engineering/software_project_management.htm

